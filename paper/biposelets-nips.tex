\documentclass{article}

% if you need to pass options to natbib, use, e.g.:
\PassOptionsToPackage{numbers,sort&compress}{natbib}
% before loading nips_2016
%
% to avoid loading the natbib package, add option nonatbib:
% \usepackage[nonatbib]{nips_2016}

% to compile a camera-ready version, add the [final] option, e.g.:
% \usepackage[final]{nips_2016}
\usepackage{nips_2016}

\usepackage[utf8]{inputenc} % allow utf-8 input
\usepackage[T1]{fontenc}    % use 8-bit T1 fonts
\usepackage{hyperref}       % hyperlinks
\usepackage{url}            % simple URL typesetting
\usepackage{booktabs}       % professional-quality tables
\usepackage{amsfonts}       % blackboard math symbols
\usepackage{nicefrac}       % compact symbols for 1/2, etc.
\usepackage{microtype}      % microtypography

\title{Human pose estimation in videos using biposelets}

% The \author macro works with any number of authors. There are two
% commands used to separate the names and addresses of multiple
% authors: \And and \AND.
%
% Using \And between authors leaves it to LaTeX to determine where to
% break the lines. Using \AND forces a line break at that point. So,
% if LaTeX puts 3 of 4 authors names on the first line, and the last
% on the second line, try using \AND instead of \And before the third
% author name.

\author{
  Sam Toyer\\
  Research School of Computer Science\\
  Australian National University\\
  Canberra, Australia 2601\\
  \texttt{u5568237@anu.edu.au} \\
  %% For more authors (see note above on \And vs. \AND)
  %% {\And, \AND}
  %% Coauthor \\
  %% Affiliation \\
  %% Address \\
  %% \texttt{email} \\
}

\begin{document}

\maketitle

\begin{abstract}
The problem of human pose estimation in still images has been well-studied in
recent years, but making effective use of the temporal information inherent in
videos is still an open problem. We propose a new model which attempts to learn
temporal relationships by using a CNN to predict poses for several frames at a
time; our model caters to the detection capabilities of CNNs by casting pose
estimation as a problem of predicting the \textit{biposelets} which poses in
adjacent frames are composed of. Predicting poses in two frames at a time
reduces pose estimation over an entire video sequence to the problem of choosing
a pose for each frame from a set of high-scoring poses, which can be achieved by
minimising a trivial pairwise cost with dynamic programming. Experiments show
that our approach performs competitively with existing approaches on established
pose estimation benchmarks.
\end{abstract}

\section{Introduction}

\section{Related work}

\section{Biposelet detection}

\section{Sequence stitching}

\section{Experiments}

\subsection{MPII Cooking}

\subsection{FLIC and PIW}

\subsection{H3.6M}

\section{Conclusion}

\section*{Acknowledgments}

\verb|citet|: \citet{tompson2014joint}

\bibliographystyle{abbrvnat}
\bibliography{citations}
\end{document}
